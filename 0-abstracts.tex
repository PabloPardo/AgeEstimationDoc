\chapter*{Abstract}

Age Estimation tackles the problem of automatically predicting the age of people from visual data (still images, video data, depth maps, etc.). 
This is a very challenging problem because ageing is affected by many factors \cite{shephard1997aging} such as bad habits, accidents, make up or facial hair.

This work focuses on three different age definitions, \textit{real age} as the actual age (time elapsed since the person was born), \textit{apparent age} as the perceived age from humans from the visual appearance and \textit{estimated age} as the predicted age by a machine from the visual appearance.

In this work, a new age estimation face image database is presented containing for first time in the literature real and apparent age annotations. A study comparing the estimation of both type of ages was done by proposing two age estimation methods using the state of the art techniques, one based on \acrfull{bif} and the other based on \acrfull{cnn}.

The results of the two methods implemented in this master thesis show the superiority of the \acrshort{cnn} over the \acrshort{bif} and also they show the difficulty of the age estimation problem with face images taken in an uncontrolled environment.

As a result of this work, an international challenge is being organized and the results will be presented in the ICCV conference 2015 (under revision).
