\chapter{Introduction} \label{chap:introduction}

Computer Vision is a field in Artificial Intelligence that has being going on since 1960 with digital image processing by computers was possible. Since the beginning of artificial vision, facial analysis has been a major interest in the research community (not just in Computer Vision but in other scientific areas such as biology \cite{bhl24064}, psychology \cite{ekm02}, neuroscience \cite{freiwald2009face} and sociology \cite{kemper1978social}) because its difficulty and its applications. Some of these applications are automatic detection of facial expressions \cite{cohen2003facial}, face detection \cite{hsu2002face}, face recognition \cite{wright2009robust}, face verification \cite{taigman2014deepface} and automatic estimation of age \cite{4359348}, gender \cite{alexandre2010gender} and ethnicity \cite{hosoi2004ethnicity}.

Age estimation is a field within the facial analysis area in Computer Vision that tackles the problem of automatically predicting the age of people from visual data (still images, video data, depth maps, etc.). One of the main issues that the age estimation problem has is that there are many factors that influence human perception of age, some factors affect the aging of a person, such as as smoking, drinking alcohol, doing sports, alimentation, etc. and others affect the face appealing such as scars, plastic surgery, make-up, facial hair, etc.

In this work, the first database in the literature of face image with real and apparent age annotations is collected and studied the difference in performance between the two different labels.

\note{Goal of the thesis}

- First database with apparent age collected

- Analysis of the two annotations (real and apparent) with two different baselines.

- Challenge

