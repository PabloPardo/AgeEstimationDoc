\chapter{Introduction} \label{chap:introduction}

Computer Vision is a very important field in Artificial Intelligence that exists since the 1960s when digital image processing by computers became possible. Since the beginning of artificial vision, facial analysis has been a major interest in the research community (not just in Computer Vision but in other scientific areas such as biology \cite{bhl24064}, psychology \cite{ekm02}, neuroscience \cite{freiwald2009face} and sociology \cite{kemper1978social}) because its difficulty and its applications. Some of its applications include automatic detection of facial expressions \cite{cohen2003facial}, face detection \cite{hsu2002face}, face recognition \cite{wright2009robust} \cite{taigman2014deepface} and automatic estimation of age \cite{4359348}, gender \cite{alexandre2010gender} and ethnicity \cite{hosoi2004ethnicity}.

Age estimation is a field within the facial analysis area in Computer Vision that tackles the problem of automatically predicting the age of people from visual data (still images, video data, depth maps, etc.). One of the main issues that the age estimation problem has is that there are many factors that influence human perception of age, some factors affect the aging of a person \cite{shephard1997aging}, such as smoking, drinking alcohol, doing sports, alimentation, etc. and others affect the face appealing such as scars, plastic surgery, make-up, facial hair, etc.

Some definitions should be established beforehand regarding the concept of human age:
\begin{itemize}
	\item \textit{Real age}: The actual age (time elapsed since the person was born).
	\item \textit{Apparent age}: Perceived age from humans from the visual appearance. 
	\item \textit{Estimated age}: The predicted age by a machine from the visual appearance.
\end{itemize}

\section{Goals}
This work aims to study the differences (if any) of automatic age estimation from real age labels and apparent age labels. Given that does not exist any face image dataset with these two label annotations, a database with such requisites has been created. In order to do so, a web-based application was developed using the \textit{Facebook API} to facilitate a collaborative and competitive collection of face images (\url{http://sunai.uoc.edu:8005/}).

As a consequence of this work, the first database in the literature containing real age and apparent age annotations for the face images was created and analysed. This database will allow researchers to tackle a different and new sub-problem of age estimation, \textit{Apparent Age Estimation}. The most important methods in the state of the art are evaluated over our proposed database in this work.

In \textit{Real Age Estimation} other external factors such as time evolution, habits and surgeries have to be taken into account. However \textit{Apparent Age Estimation} is based purely in the perception field.


\section{Age Challenge}

Given the innovative aspect of the database, the HuPBA (Human Pose and Behaviour Analysis) research group\footnote{Human Pose and Behaviour Analysis research group\\ \url{http://www.maia.ub.es/~sergio/soluciones2_008.htm}} together with the non-profit organization  ChaLearn\footnote{ChaLearn: Challenges in Machine Learning: \url{http://www.chalearn.org/}} are going to prepare an international challenge competition later this year and present the results in a workshop organized by the team in the ICCV conference 2015 edition (under revision) within the ChaLearn Looking at People series \cite{LaP} \cite{SergioEscalera2014} \cite{Escalera:2013:MGR:2522848.2532595} \cite{conf/icmi/EscaleraGBRGAESASBS13}. The challenge and the workshop will be sponsored by companies such as Google, Microsoft Research, Amazon and International Association for Pattern Recognition (IAPR) among others.

The challenge pretends to establish State of the Art techniques for Apparent Age Estimation and compare the methods and the results with the Real Age Estimation State of the Art.

Initial results regarding the database and the methods presented in this work have been published in the International Joint Conference on Neural Networks (IJCNN).


