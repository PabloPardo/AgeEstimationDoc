\chapter{Introduction} \label{chap:introduction}

Computer Vision is a field in Artificial Intelligence that has being going on since 1960 with digital image processing by computers was possible. Since the beginning of artificial vision, facial analysis has been a major interest in the research community (not just in Computer Vision but in other scientific areas such as biology \cite{bhl24064}, psychology \cite{ekm02}, neuroscience \cite{freiwald2009face} and sociology \cite{kemper1978social}) because its difficulty and its applications. Some of these applications are automatic detection of facial expressions \cite{cohen2003facial}, face detection \cite{hsu2002face}, face recognition \cite{wright2009robust}, face verification \cite{taigman2014deepface} and automatic estimation of age \cite{4359348}, gender \cite{alexandre2010gender} and ethnicity \cite{hosoi2004ethnicity}.

Age estimation is a field within the facial analysis area in Computer Vision that tackles the problem of automatically predicting the age of people from visual data (still images, video data, depth maps, etc.). One of the main issues that the age estimation problem has is that there are many factors that influence human perception of age, some factors affect the aging of a person \cite{shephard1997aging}, such as smoking, drinking alcohol, doing sports, alimentation, etc. and others affect the face appealing such as scars, plastic surgery, make-up, facial hair, etc.

Some definitions should be established beforehand regarding the concept of human age:
\begin{itemize}
	\item \textit{Real age}: The actual age (number of years passed since the person was born).
	\item \textit{Apparent age}: Perceived age from humans from the visual appearance. 
	\item \textit{Estimated age}: The predicted age by a machine from the visual appearance.
\end{itemize}

This work aims to create and analyse  the first database in the literature containing real age and apparent age annotations for the face images. In order to do so, a web-based application has been developed using \textit{Facebook API} to facilitate a collaborative and competitive collection of images.

\note{Analysis of the two annotations (real and apparent) with two different baselines.}

Most of the related work in the literature are based on estimating real age. However, in this work, apparent age will also be studied

\note{Challenge}

Given the innovative of the database, ChaLearn is going to prepare a challenge for the ICCV conference, challenging the participants to upload better solutions to the problem than the ones presented here.
