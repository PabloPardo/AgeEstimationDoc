\chapter{State of the art} \label{chap:sota}

Age estimation has historically been one of the most challenging problems within the field of facial analysis \cite{5406526}\cite{han:age}. Despite the multiple applications in many different areas of age estimation there are relatively few publications compared to other topics in facial analysis. This difficulty is due to many factors: 
\begin{itemize}
	\item Depending on the application scenario, the age estimation problem can be taken as a multiclass classification problem or a regression problem.
	\item Large databases are difficult to collect, especially series of chronological image from the same individuals.
	\item The factors the affect the ageing process are uncontrollable and person specific \cite{4284917}\cite{4359348}\cite{1709980}.
\end{itemize}

The age estimation problem has generally two stages or blocks, the first one is the age representation in the images and the second one the learning algorithm. Several techniques have been published in order to deal with these two stages \cite{5406526}.

\section{Age Representation}

Age Representation is a very important step in the age estimation process. A good age representation will contain enough variation of the data to express the full complexity of the problem. There are many ways in the literature to represent ageing factors from an image. The most influential ones are described next.

\subsection{Anthropomorphic Models}
The first known work on age estimation from facial images was done by Y. Kwon and N. Lobo \cite{Kwon:1999:ACF:311844.311845}. Their approach is based in cranio-facial development theory using geometrical ratios between different face regions to classify images into one of three age groups (babies, young adults and senior adults). They used frontal images in a very strict set up to be able to locate all face components. N. Ramanathan et al. \cite{1640784, Ramanathan2009131} used a similar approach in this case using 8 ratios rather than the 6 used by Y. Kwon et al.

The problem of this model is that only can be applied to face images of young people in a growing age since afterwards the facial geometry does not change as much. It is also a problem that the methods require of frontal images since it limits the future applications.

\subsection{Active Appearance Models}

Active Appearance Models (AAM) is a statistical shape model proposed by T.F Cootes et al. \cite{Cootes:2001:AAM:378040.378090}. This model contains the shape and grey-level appearance of the object of interest which can generalize to almost any valid example. This technique has been used to find the shape of faces by many researchers. A. Lanitis et al. \cite{791208, 993553, Lanitis:2004:CDC:2225304.2226166} were the first in extend the AAM model for age estimation by defining an ageing function $age=f(b)$, where \textbf{f} is an ageing function and \textbf{b} is a vector containing the parameters learned by the AAM.

A. Lanitis et al. \cite{Lanitis:2004:CDC:2225304.2226166} also tried different classifiers such as Quadratic Functions, Shortest Distance Classifier, Supervised Neural Network and Unsupervised Neural Network. Among all of them they reported that Quadratic Functions where the ones performing best.

Later on, K. Luu et al. \cite{Luu:2009:AEU:1736406.1736456, LuuSSBS11}, use in \cite{Luu:2009:AEU:1736406.1736456} AAM with 68 facial landmarks to classify faces into young or adult classes and then use two specialised functions to finally determine the age. Luu's approach performs better than Lanitis' one \cite{Lanitis:2004:CDC:2225304.2226166}. In \cite{LuuSSBS11} improves further the previous method by proposing Contourlet Appearance Models (CAM) which is more accurate and faster at calculate facial landmarks than AAM. This model has the ability not to just capture global texture information like AAM but also local texture information using Nonsubsampled Contourlet Transform (NSCT) \cite{1703596}.

Chang et al. in \cite{5995437} used AAM model with particular ranking formulation of support vectors, OHRank. The approach uses cost-sensitive aggregation to estimate ordinal hyperplanes (OH) and ranks them according to the relative order of ages.

This model captures shape and texture information and in general performs better than the \textit{Anthropomorphic Models}. This method can deal with any range of ages rather than just with young ages like the previous model. However, as suggested by X. Geng et al. \cite{Geng:2006:LFA:1180639.1180711}, the ageing functions is empirically determined, so there is no evidence suggesting that the relation between face and age is described just by a quadratic function.

\subsection{Ageing Pattern Subspace}

X. Geng et al. \cite{4359348, Geng:2006:LFA:1180639.1180711} were the ones that explored this model initially which is called AGing pattErn Subspace (AGES). They define an \textit{ageing pattern} as a sequence of personal face images sorted in time order. Given a grey-scale face image $\textbf{I}$, where $\textbf{I}(x,y)$ determines the intensity of the pixel $(x,y)$, then an ageing pattern can be represented as a three-dimensional matrix $\textbf{P}$, where $\textbf{P}(x,y,t)$ is the intensity of the pixel $(x,y)$ in the face image at the time $t$. The images vector is filled with the available face images leaving empty the missing faces in the $t$ axis. Now, the images in the age pattern vector can be precessed and transformed into meaningful feature vectors.

In order to extract the features X. Geng use AAM as used in \cite{791208} since they capture the shape and texture of the face images. By representing ageing patterns in this way, the concepts of identity and time are naturally integrated into the data without any pre-assumptions.

The principal drawback of the AGES method is that assumes that there are images of the same individual at different ages, which is not true in all the age databases, like in the Yamaha Gender and Age (YGA) database \cite{4523958}, and it is difficult to collect such a databases.

\subsection{Age Manifold}
 
The manifold learning methods are applied to find a sufficient embedding space and model the low-dimensional manifold data with a multiple linear regression function. Y. Fu et al. \cite{4523958, 4284917} were the first in proposing a manifold embedding approach for the age estimation problem. 

The objective of this method is to find the low-dimensional representation in the embedded subspace capturing the intrinsic data distribution and geometric structure as well as its representation. G. Guo et al. \cite{Guo:2008:IHA:2319085.2321608} \cite{4531189} shows that the Orthogonal Locality Preserving Projections (OLPP) \cite{CHHZ06}  is a good an effective algorithm  to connect the manifold learning with subspace learning. In a posterior work \cite{5995404}, G. Guo et al. introduces a new approach, using kernel partial least square (KPLS) regression which reduces feature dimensionality and learn the ageing function in a single step.

Then T. Wu et al. \cite{journals/tifs/WuTC12} proposed to model the facial shapes as points on a Grassmann manifold. Age estimation is then considered as regression and classification problems on this manifold. Then, they proposed a method for combining this shape-based approach with other texture-based algorithms.

The main drawback of the age manifold representation is the large number of training instances required to learn the embedded manifold with statistical sufficiency.

\subsection{Appearance Models}

Appearance Models focus on wrinkles, face texture and pattern analysis. From the beginning, researchers have tried to capture wrinkles and distinguish them from facial lines. Y. Kwon et al. \cite{Kwon:1999:ACF:311844.311845} proposed a wrinkle detector based on snakelets \cite{Kass88snakes:active} placed into key wrinkle areas of the face. Hayashi et al. \cite{969698} \cite{1195171} \cite{conf/icpr/HayashiYIK02} combined both shape and texture to estimate age and gender. In Hayashi's proposed approach, the skin is extracted based on a shape model and then a histogram equalization is applied to emphasize wrinkles.

Other researchers have used the texture descriptor Local Binary Patterns (LBP) \cite{Ahonen:2006:FDL:1175897.1176245} in the age estimation problem, such as \cite{4717926} \cite{6460367} obtaining good classification results with Nearest Neighbour and SVM classification algorithms. The Gabor \cite{Liu:2002:GFB:2319007.2320264} filter texture descriptor has also been used in the age estimation task  \cite{Gao:2009:FAC:1567988.1568003}, probing to be more discriminative than LBP.

G. Guo et al. \cite{conf/cvpr/GuoMFH09} proposed to use Biological Inspired Features (BIF) \cite{Riesenhuber99hierarchicalmodels} for age estimation via faces. The BIF descriptor tries to mimic how the visual cortex works, with a hierarchy of increasingly sophisticated representations. The BIF original model proposed by Riesenhuber and Poggio \cite{Riesenhuber99hierarchicalmodels} is based in a feed-forward model of the primate visual object recognition pathway, the ``HMAX'' model. The framework of the model contains alternative layers called simple (S) and complex (C) creating in each cycle a more elaborated representation. The S layers are created with a Gabor filtering on the input and the C layers generally operates a ``MAX'' operator over the previous S layer. G. Guo et al. \cite{conf/cvpr/GuoMFH09} modifies tha BIF model by changing the operator in the complex layer (C1) from ``MAX'' to ``STD''.

Han et al. \cite{han:age} uses the BIF features in an hybrid classification framework improving the previous results with this descriptor. G. Guo et al. \cite{Guo2014761}, in a recent paper (2014), also used the BIF features, and focus to investigate a proposed single-step framework for joint estimation of age, gender and ethnicity. Both the CCA (Canonical Correlation Analysis) \cite{hotelling1936relations} and PLS (Partial Least Square) based methods were explored under the joint estimation framework.

In \cite{6553772}, Weng et al. employs a similar ranking technique than Chang et al. in \cite{5995437} called MFOR. Local Binary Patterns (LBP) histogram features are combined with principal components of BIF, shape and textural features of AAM, and PCA projection of the original image pixels. Fusion of texture and local appearance descriptors (LBP and HOG features) have independently also been used for age estimation by Huerta et al. in \cite{HuFerPra14}.

The outstanding results obtained by some of these works point out the suitability of the BIF features for the age estimation via faces task.

\subsection{Other Models}

Depending on the available data the age estimation problem changes. Many researchers have tackled the age estimation problem with different types of data, a short overview is described below.

N. Ramanathan et al. \cite{1709980} studied age progression of individual faces and proposed a method to perform face verification using a Bayesian age-difference classifier to improve the face verification algorithm.

A. Lanitis et al. \cite{5463396} collected a database of head and mouse tracking movements and then attempted to perform age estimation with this data. This preliminary study shows the potential of using this type of data.

Y. Mikihara et al. \cite{6117531} used a gait-based database to perform a viability study of age estimation using this type of data. The results show that in future research the combination of gait-based data and face-based age estimation could give very good results.

B. Xia et al. \cite{xia:hal-00904007} were the first to attempt age estimation using 3D face images. The obtained results show that the depth dimension has a very discriminative power in this problem. 

\section{Age Estimation Learning Algorithm}
Given an age representation, the next step is to determine the individual's age out of the ageing features. Age labels can be seen as a discrete set of classes or as a continuous label space, hence classification and regression learning methods can be used.

\subsection{Classification Methods}

The age estimation problem can be treated as a classification problem, where solution space is discrete and the objective is to classify each face image into one of the age classes (a class could be an age range of several years or a single year).

A. Lanitis et al. \cite{Lanitis:2004:CDC:2225304.2226166} evaluated the performance of different classifiers such as quadratic function classifier, Artificial Neural Network (ANN) and nearest neighbour classifier with their AAM model. Among the classification methods they test, they claimed to perform better with the ANN, specifically the Multi Layer Perceptron (MLP), obtaining 4.78 MAE. The authors also proposed some extensions, for example, train age specific classifiers in a hierarchical fashion. With the extended methods the authors reduced the error to 4.38 MAE with the MLP and 3.82 MAE with the quadratic function (regression).

There have been previous proposals training neural networks,
which are able to learn complex mappings and deal with
outliers, for age estimation. In \cite{Lanitis:2004:CDC:2225304.2226166}, Lanitis et al. used AAM encoded face parameters as an input for the supervised training
of a neural network with a hidden layer. More recently, Geng
et al. in \cite{6475129} tackle age estimation as a discrete classification problem using 70 classes, one for each age. The best algorithm proposed in this work (CPNN - Conditional Probability Neural Network) consists of a three-layered neural network, in which the input to the network includes both BIF features $x$ and a numerical value for age $y$, and the output neuron is a single value of the conditional probability density function $p(y|x)$. An extensive comparison of these classification schemes for age estimation has been reported in Fernandez et al. \cite{2014icprw}. In \cite{5995481}, Yang et al. used Convolutional neural networks for age estimation under surveillance scenarios.

K. Ueki \cite{Ueki:2006:SAC:1126250.1126269} classified the images from the WIT\_DB database into 11 age groups using Gaussian models in a low-dimensional 2DLDA+LDA feature space using the EM Algorithm. The accuracy rates they achieved were $46.3\%$, $67.8\%$ and $78.1\%$ for age groups that were in the 5-year, 10-year and 15-year range respectively.

SVM have been also used for age classification, Guo et al. \cite{4544009} \cite{Guo:2008:IHA:2319085.2321608} trained an SVM for each pair of age classes and then using a binary tree search for testing, obtaining 5.55 MAE for females and 5.52 MAE for males in the YGA database and 7.16 MAE in the FG-NET database. 


\subsection{Regression Methods}

The age of an individual is nothing else than the time passed from the individual's birth, and time is a continuous dimension. Hence, the age estimation problem can be formulated as a regression problem where the objective is to find a regression function that explains the ageing in terms of the feature space.

A. Lanitis et al. \cite{Lanitis:2004:CDC:2225304.2226166} evaluated three regression functions, linear, quadratic and cubic and claimed that the quadratic function was the one which better described the age from their feature space. Y. Fu et al. \cite{4523958} \cite{4284917} used linear, quadratic and cubic regression function as a learning algorithms from the manifold age representation. As Lenitis showed, Fu also report superior performance on the quadratic regression function, pointing out that cubic functions lead to over-fitting while linear functions lead to under-fitting.

Guo et al. \cite{4544009} \cite{Guo:2008:IHA:2319085.2321608} compared Support Vector Regressor (SVR) with Local Adjusted Robust Regression (LARR) performance in the YGA and the FG-NET databases, concluding that LARR performs a more accurate estimation, achieving 5.25 MAE in YGA for female images, 5.30 in YGA for male images and  5.07 in FG-NET database. In a posterior work \cite{conf/cvpr/GuoMFH09}, the authors improve the SVR performance in the FG-NET to 4.77 MAE by using BIF features.

\subsection{Hybrid Methods}

Many authors have proposed mixture frameworks, using both classification and regression learning algorithms. 

G. Guo et al. \cite{4563041} proposed a probabilistic fusion approach. They use Bayes' rule to derive the predictor and then a sequential fusion strategy, so the output of the regressor is used as an intermediate decision which is then fed to the classifier to aid or affect the decision space of the classifier. Their fusion approach has better performance than other single step methods which they compare with.

SVM and SVR where used by Han et al. \cite{han:age} in a hierarchical fashion. They proposed to use a binary decision tree with SVMs at each node to classify the images into different age ranges, which are coarsely assigned. Later, the age is fine grained by SVRs at the leaves. The SVRs are trained with 5 years overlapping between age ranges in order to reduce the misclassification error.


\begin{table}[h!]
	\centering
	\resizebox{\textwidth}{!}{
		\begin{tabular}{|l|c|M{3cm}|M{3cm}|M{3cm}|M{3.5cm}|N}
			\cline{1-7}
			\rowcolor[HTML]{EFEFEF} 
			\hline
			\textbf{Publication} & 
			\textbf{Year} &
			\textbf{Database (\#subjects, \#images)} &
			\textbf{Age Image Representation} &
			\textbf{Learning Method} & 
			\textbf{MAE} &\\[8pt] \hline
			
			\multicolumn{1}{|p{2.5cm}|}{\textbf{A. Lanitis et al. \cite{993553}}} & 2002 & Private (60, 500) & AAM & Quadratic Aging Function & $3.94\pm3.8$&\\[5pt] \hline
			
			\multicolumn{1}{|p{2.5cm}|}{\textbf{A. Lanitis et al. \cite{Lanitis:2004:CDC:2225304.2226166}}} & 2004 & Private (40, 400) & AAM & Quadratic Aging Function & $3.82\pm5.58$&\\[5pt] \hline
			
			\multicolumn{1}{|p{2.5cm}|}{\textbf{X. Geng et al. \cite{Geng:2006:LFA:1180639.1180711}}} & 2006 & FG-NET & AGES & Regression & $6.77$ &\\[5pt] \hline
			
			\multicolumn{1}{|p{2.5cm}|}{\textbf{Y. Fu et al. \cite{4284917}}} & 2007 & YGA & Manifold & Linear regression function& $5\sim 6$ &\\[5pt] \hline
			
			\multicolumn{1}{|p{2.5cm}|}{\textbf{G. Guo et al. \cite{conf/cvpr/GuoMFH09}}} & 2009 & FG-NET, YGA & BIF & SVR & FG-NET/YGA $4.77$/F:$3.91$, M:$3.47$ &\\[5pt] \hline
			
			\multicolumn{1}{|p{2.5cm}|}{\textbf{K. Luu et al. \cite{Luu:2009:AEU:1736406.1736456}}} & 2009 & FG-NET & AAM & Hierarchical framework (SVM, SVR) & $4.37$ &\\[5pt] \hline
			
			\multicolumn{1}{|p{2.5cm}|}{\textbf{K. Luu et al. \cite{LuuSSBS11}}} & 2011 & FG-NET, PAL & Contourlet Appearance Model (CAM) & Hierarchical framework (SVM, SVR) & FG-NET/PAL $4.12$ / $6.0$ &\\[5pt] \hline
			
			\multicolumn{1}{|p{2.5cm}|}{\textbf{G. Guo et al. \cite{5995404}}} & 2011 & MORPH-II & BIF & Kernel PLS Regression & $4.18$ &\\[5pt] \hline
			
			\multicolumn{1}{|p{2.5cm}|}{\textbf{H. Han et al. \cite{han:age}}} & 2013 & FG-NET, MORPH-II, PCSO & BIF + Active Shape Models & Hierarchical framework (SVM, SVR) &  FG-NET/PCSO $4.7\pm24.8$ / $7.2\pm32.0$ &\\[5pt] \hline
			
			\multicolumn{1}{|p{2.5cm}|}{\textbf{G. Guo et al. \cite{Guo2014761}}} & 2014 & MORPH-II & BIF & rKCCA\footnote{Regularized Kernel Canonical Correlation Analysis} + Linear SVM & $3.92$ &\\[5pt] \hline
			
			\multicolumn{1}{|p{2.5cm}|}{\textbf{I. Huerta et al. \cite{HuFerPra14}}} & 2014 & FG-NET, MORPH-II & BIF + HOG + LBP + SURF + Gradient & rCCA\footnote{Regularized Canonical Correlation Analysis} & FG-NET/MORPH $4.17$ / $4.25$ &\\[5pt] \hline

		\end{tabular}
	}
	\caption{Age Estimation Methods}
	\label{tab:age-methods}
\end{table}

\section{Applications}
There are many real-world application related to age estimation. Automatic age estimation is useful in situations where there is no need to specifically identify the individual, such as a government employee, but want to know his or her age.

\subsection{Security Control and Surveillance Monitoring}
In the last years security control and surveillance monitoring have gotten more relevant with the growth of internet content and the spread of technology that allows access to that content to under-age teenagers. Automatic age estimation systems can be used to prevent minors to buy alcohol in a grocery store, enter a bar or purchase tobacco from vending machines.

\subsection{Biometrics}
There are two types of biometric systems based on the number of traits used for recognition, unimodal biometric systems which consist on a single recognition trait and multimodal biometric systems, which combines evidences obtained from multiple sources \cite{MSU-CSE-99-39} such as fingerprints, iris, face, etc. The multimodal system is more robust, more reliable and secure against spoof attacks. However, the data acquisition is much more troublesome than the unimodal. In order to overcome this inconveniences, soft-biometrics \cite{conf/icba/JainDN04}, such as age, hight, weigh, gender, ethnicity and eye colour, are used in combination with classic biometric traits. 

\subsection{Age-based Indexing Face Databases}
With the rise of interest for big data new and more efficient ways to retrieve data have to be developed. In large face image datasets, age can be used for index such a databases so the queries to the dataset are simpler and faster. This is specially important in law enforcement where large image databases of suspects have to be filtered in order to find the most accurate suspects.

\subsection{Human-Computer Interaction and e-Commerce}
With the growth of e-commerce, companies want to offer a more personalized experience to their customers. Personalizing the offer or the product itself increase the user's satisfaction and the companies sells. Some examples of such a policies are the following: Google \cite{Brin:1998:ALH:297810.297827} indexes the search results so the links that appear first appeal more to the user, Amazon \cite{Linden:2003:ARI:642462.642471} uses a recommender system to suggest products to the potential buyers according to their previous purchases, Netflix \cite{Koren:2009:MFT:1608565.1608614} held a competition in 2009 to create a film recommender system and gave a price of US \$1,000,000. Age estimation system could have an important role in the sector since age is a discriminative feature for different client profiles. Visada \cite{visada} is an example of the use of age estimation for recommend products.


\section{Age-based Datasets} \label{sec:ageDB}
There are many databases of faces in the literature, however, not so many capture the age of the individuals. This fact is due to the complexity of crawling such an information (if existent) from the usual fonts such as \textit{Flickr} or \textit{Facebook} and due to privacy issues. Moreover, the difficulty is even higher if the database contains chronological image series of individuals. The Table \ref{tab:age-db} shows the most relevant databases used in the literature with the number of samples, the number of subjects, the age range, the type of age annotations and additional information if any. The \textit{FG-NET} \cite{993553} is one of the first and most consolidated age database, it is used to compare with other age estimation methods.

After an initial interest in automatic age estimation from images dated back to the early 2000s \cite{Lanitis:2004:CDC:2225304.2226166}, \cite{993553}, \cite{palDB}, research in the field has experienced a renewed interest from 2006 on, since the availability of large databases like \textit{MORPH-Album 2} \cite{1613043}, which contains 55 times more age-annotated images than the \textit{FG-NET} database.

\begin{table}[t!]

\centering
\resizebox{\textwidth}{!}{

\begin{tabular}{|l|c|c|c|c|M{2.5cm}|M{2.5cm}|M{2.5cm}|N}
	\hline
	\cline{1-8}
	\hline
	\cellcolor[HTML]{EFEFEF}\textbf{Database} & \cellcolor[HTML]{EFEFEF}\textbf{\#Faces} & \cellcolor[HTML]{EFEFEF}\textbf{\#Subj.} & \cellcolor[HTML]{EFEFEF}\textbf{Range} & \cellcolor[HTML]{EFEFEF}\textbf{Type of age} & \cellcolor[HTML]{EFEFEF}\textbf{Controlled Env.} & \cellcolor[HTML]{EFEFEF}\textbf{Balanced age Distr.} & \cellcolor[HTML]{EFEFEF}\textbf{Other annotation} &\\[8pt] \hline
	
	\multicolumn{1}{|p{2cm}|}{\textbf{FG-NET \cite{993553, FGNET}}} & 1,002 & 82 & 0 - 69 & Real Age & No & No & 68 Facial Landmarks &\\[5pt] \hline
	
	\multicolumn{1}{|p{2cm}|}{\textbf{GROUPS \cite{gallagher_cvpr_09_groups}}} & 28,231 & 28,231 & 0 - 66+ & Age group & No & No & - &\\[5pt] \hline
	
	\multicolumn{1}{|p{2cm}|}{\textbf{PAL \cite{palDB}}} & 580 & 580 & 19 - 93 & Age group & No & No & - &\\[5pt] \hline
	
	\multicolumn{1}{|p{2cm}|}{\textbf{FRGC \cite{frgcDB}}} & 44,278 & 568 & 18 - 70 & Real Age & Partially & No & - &\\[5pt] \hline
	
	\multicolumn{1}{|p{2cm}|}{\textbf{MORPH2 \cite{1613043}}} & 55,134 & 13,618 & 16 - 77 & Real Age & Yes & No & - &\\[5pt] \hline
	
	\multicolumn{1}{|p{2cm}|}{\textbf{YGA \cite{4523958}}} & 8,000 & 1,600 & 0 - 93 & Real Age & No & No & - &\\[5pt] \hline
	
	\multicolumn{1}{|p{2cm}|}{\textbf{FERET\cite{Phillips1998295}}} & 14,126 & 1,199 & - & Real Age & Partially & No & - &\\[5pt] \hline
	
	\multicolumn{1}{|p{2cm}|}{\textbf{Iranian face \cite{4469272}}} & ~3,600 & 616 & 2 - 85 & Real Age & No & No & Kind of skin and cosmetic points \footnote{Surgical points, fracture or laceration on face.}  &\\[5pt] \hline
	
	\multicolumn{1}{|p{2cm}|}{\textbf{PIE \cite{1004130}}} & 41,638 & 68 & - & Real Age & Yes & No & - &\\[5pt] \hline
	
	\multicolumn{1}{|p{2cm}|}{\textbf{WIT-BD \cite{Ueki:2006:SAC:1126250.1126269}}} & 26,222 & 5,500 & 3 - 85 & Age group & No & No & - &\\[5pt] \hline
	
	\multicolumn{1}{|p{2cm}|}{\textbf{Caucasian Face Database \cite{burt1995perception}}} & 147 & - & 20 - 62 & Real Age & Yes & No & Shape represented in 208 key points &\\[5pt] \hline
	
	\multicolumn{1}{|p{2cm}|}{\textbf{LHI \cite{LHI}}} & 8,000 & 8,000 & 9 - 89 & Real Age & Yes & Yes & - &\\[5pt] \hline
	
	\multicolumn{1}{|p{2cm}|}{\textbf{HOIP \cite{HOIP}}} & 306,600 & 300 & 15 - 64 & Age Group & Yes & No & - &\\[5pt] \hline
	
	\multicolumn{1}{|p{2cm}|}{\textbf{Ni's Web-Collected Database \cite{Ni:2009:WIM:1631272.1631287}}} & 219,892 & - & 1 - 80 & Real Age & No & No & - &\\[5pt] \hline
	
	\multicolumn{1}{|p{2cm}|}{\textbf{OUI-Adience \cite{6906255}}} & 26.580 & 2.284 & 0 - 60+ & Age Group & No & No & Gender &\\[5pt] \hline

\end{tabular}

}

\caption{Age-based Databases}
\label{tab:age-db}

\end{table}
