\chapter{State of the art} \label{chap:sota}

\section{Historical context}
One of the earliest works in age estimation was done by A. Lanitis et al. \cite{Lanitis:2004:CDC:2225304.2226166, 791208, 993553} where the age is modelled by a quadratic aging function. They propose two different aging estimation methods: weighted appearance specific method \cite{791208, 993553}  -where the aging factor of a new individual is computed by the weighted sum of the aging functions of other individual- and appearance and age specific method \cite{Lanitis:2004:CDC:2225304.2226166} -where the new individual is first classified into a cluster with similar aging factor patterns, then is classified into different age ranges and an age-specific classifier is applied to estimate the final age-. 

There are some drawbacks to this "aging function" approach pointed out by X. Geng et al. \cite{Geng:2006:LFA:1180639.1180711}. The formula of the aging function is empirically determined, there is no evidence suggesting that the relation between face and age is described just by a quadratic function. The new aging function for the unseen images is simply a linear combination of the already known aging functions. X. Geng et al. claimed to solve these problems in their new proposed method AGing pattErn Subspace (AGES) \cite{Geng:2006:LFA:1180639.1180711}. Where each face image is represented by a point in the aging pattern subspace.

Later N. Ramanathan et al. \cite{1709980, 1640784} approached the age estimation problem in two different scenarios, estimating the age difference between two face images of the same individual based on a Bayesian age-difference classifier \cite{1709980} and estimating the age of young faces using the facial growth geometry \cite{1640784}. The problem of the last approach is that only can be applied to face images of young people in a growing age since afterwards the facial geometry does not change as much.

Y. Fu et al. were the first on approach the problem through manifold analysis methods \cite{4284917, 4523958}. Each face image is assigned to its low-dimensional representation via manifold embedding. Following this approach G. Guo and Y. Fu et al. \cite{4531189} proposed a new method based on a study of different dimensionality reduction and manifold embedding and add a robust regression step to the previous framework. In a posterior work \cite{5995404}, G. Guo et al. introduces a new approach, using kernel partial least square (KPLS) regression which reduces feature dimensionality and learn the aging function in a single step.

G. Guo et al. also proposed different approaches to the age estimation problem such as
\cite{4563041}, where they propose probabilistic fusion approach, or \cite{conf/cvpr/GuoMFH09} where they introduce the Biological Inspired Features (BIF) for the age estimation problem and propose some changes adding a novel "STD" operator. H. Han et al. \cite{han:age} uses the BIF features in an hybrid classification framework improving the previous results. G. Guo et al. \cite{Guo2014761}, in a recent paper (2014), used the BIF features, and focus to investigate a proposed single-step framework for joint estimation of age, gender and ethnicity. Both the CCA (Canonical Correlation Analysis) and PLS (Partial Least Square) based methods were explored under the joint estimation framework.

Under the same idea as Y. Fu et al. \cite{4284917}, K. Luu et al. \cite{Luu:2009:AEU:1736406.1736456, LuuSSBS11} reduced dimensionality by using facial landmarks and Active Shape Models (ASM) \cite{Luu:2009:AEU:1736406.1736456} and an improved version, Contourlet Appearance Model (CAM) \cite{LuuSSBS11}, where they prove the efficiency of using facial landmarks. Then T. Wu et al. \cite{journals/tifs/WuTC12} proposed to use facial landmarks and project them into a Grassmann manifold to model the age patterns.

Other different variations of the problem has been addressed, A. Lanitis et al. \cite{5463396} performed a first approach to age estimation using Head and Mouse tracking movements, Y. Makihara et al. \cite{6117531} used a gait-based database to estimate the age, B. Xia et al. \cite{xia:hal-00904007} proposed an age estimation method based on 3D face images.

There are also some surveys in age estimation by N. Ramanathan et al. \cite{Ramanathan2009131} and Y. Fu et al. \cite{5406526}.

More recently studies have obtained a very high accuracy in face validation using deep learning \cite{facebook}.


\section{Applications}
Age estimation has always been a topic of interest because of its applications, however its a difficult task and it was not until few years ago that the community started getting useful results. The main applications for age estimation methods are the following:

\subsection{Safety and control}
Automatic age estimation systems can be used to prevent users from a certain age to access banned products.
\subsection{Age-based indexing face images}
The systems can be used to give a more personalised service to the user.
\subsection{Human-computer interaction}
Indexing large face images datasets for a fast an easy retrieval of faces is very important in for example law enforcement (filtering suspects images).


\section{Age-based Databases}
There are many databases of faces in the literature, however, not so many capture the age of the individuals. This fact is due to the complexity of crawling such an information (if existent) from the usual fonts such as \textit{Flickr} or \textit{Facebook} and due to privacy issues. Moreover, the difficulty is even higher if the database contains chronological image series of individuals. The Table \ref{tab:age-db} shows the most relevant databases used in the literature. 
 
\begin{table}[h!]
	\centering
	\resizebox{\textwidth}{!}{
		\begin{tabular}{l|c|c|c|c|M{2.5cm}|M{2.5cm}|M{2.5cm}|N}
			\cline{1-8}
			\cellcolor[HTML]{EFEFEF}\textbf{Database} & \cellcolor[HTML]{EFEFEF}\textbf{\#Faces} & \cellcolor[HTML]{EFEFEF}\textbf{\#Subj.} & \cellcolor[HTML]{EFEFEF}\textbf{Range} & \cellcolor[HTML]{EFEFEF}\textbf{Type of age} & \cellcolor[HTML]{EFEFEF}\textbf{Controlled Env.} & \cellcolor[HTML]{EFEFEF}\textbf{Balanced age Distr.} & \cellcolor[HTML]{EFEFEF}\textbf{Other annotation} &\\[8pt] \hline
			\multicolumn{1}{|p{1.8cm}|}{\textbf{FG-NET \cite{FGNET}}} & 1.002 & 82 & 0 - 69 & Real Age & No & No & 68 Landmarks &\\[5pt] \hline
			\multicolumn{1}{|p{1.8cm}|}{\textbf{Morph2 \cite{1613043}}} & 55.134 & - & 16 - 77 & Real Age & Yes & No & - &\\[5pt] \hline
			\multicolumn{1}{|p{1.8cm}|}{\textbf{YGA \cite{4523958}}} & 8.000 & 1.600 & 0 - 93 & Real Age & No & No & - &\\[5pt] \hline
			\multicolumn{1}{|p{1.8cm}|}{\textbf{FERET \cite{Phillips1998295}}} & 14.126 & 1.199 & - & Real Age & Partially & No & - &\\[5pt] \hline
			\multicolumn{1}{|p{1.8cm}|}{\textbf{Iranian face \cite{4469272}}} & ~3.600 & 616 & 2 - 85 & Real Age & No & No & Kind of skin and cosmetic points\footnote{Surgical points, fracture or laceration on face.}  &\\[5pt] \hline
			\multicolumn{1}{|p{1.8cm}|}{\textbf{PIE \cite{1004130}}} & 41.638 & 68 & - & Real Age & Yes & No & - &\\[5pt] \hline
			\multicolumn{1}{|p{1.8cm}|}{\textbf{Images of Gropus \cite{gallaghercvpr09groups}}} & 28.231 & - & 0 - 66+ & Age group & No & No & - &\\[5pt] \hline
			\multicolumn{1}{|p{1.8cm}|}{\textbf{WIT-BD \cite{Ueki:2006:SAC:1126250.1126269}}} & 26.222 & 5.500 & 3 - 85 & Age group & No & No & - &\\[5pt] \hline
			\multicolumn{1}{|p{1.8cm}|}{\textbf{Caucasian Face Database \cite{burt1995perception}}} & 147 & - & 20 - 62 & Real Age & Yes & No & Shape represented in 208 key points &\\[5pt] \hline
			\multicolumn{1}{|p{1.8cm}|}{\textbf{LHI \cite{LHI}}} & 8.000 & 8.000 & 9 - 89 & Real Age & Yes & Yes & - &\\[5pt] \hline
			\multicolumn{1}{|p{1.8cm}|}{\textbf{HOIP \cite{HOIP}}} & 306.600 & 300 & 15 - 64 & Age Group & Yes & No & - &\\[5pt] \hline
			\multicolumn{1}{|p{1.8cm}|}{\textbf{Gallagher’s Web-Collected Database \cite{gallaghercvpr09groups}}} & 28.231 & - & 0 - 66+ & Age Group & No & No & - &\\[5pt] \hline
			\multicolumn{1}{|p{1.8cm}|}{\textbf{Ni’s Web-Collected Database \cite{Ni:2009:WIM:1631272.1631287}}} & 219.892 & - & 1 - 80 & Real Age & No & No & - &\\[5pt] \hline
			
			\multicolumn{1}{|p{1.8cm}|}{\textbf{OUI-Adience \cite{6906255}}} & 26.580 & 2.284 & 0 - 60+ & Age Group & No & No & Gender &\\[5pt] \hline
			
		\end{tabular}
	}
	\caption{Age-based Databases}
	\label{tab:age-db}
\end{table}

\begin{table}[h!]
	\centering
	\resizebox{\textwidth}{!}{
		\begin{tabular}{|l|c|M{3cm}|M{3cm}|M{3cm}|c|c|N}
			\cline{1-7}
			\cellcolor[HTML]{EFEFEF}\textbf{Publication} & \cellcolor[HTML]{EFEFEF}\textbf{Year} &\cellcolor[HTML]{EFEFEF}\textbf{Database (\#subjects, \#images)} & \cellcolor[HTML]{EFEFEF}\textbf{Age Image Representation} & \cellcolor[HTML]{EFEFEF}\textbf{Method} & \cellcolor[HTML]{EFEFEF}\textbf{Accuracy} & \cellcolor[HTML]{EFEFEF}\textbf{MAE} &\\[8pt] \hline
			\multicolumn{1}{|p{2.5cm}|}{\textbf{A. Lanitis et al. \cite{993553}}} & 2002 & Private (60, 500) & Active Appearance Models & Quadratic Aging Function & 71$\%$ & $3.94\pm3.8$&\\[5pt] \hline
			\multicolumn{1}{|p{2.5cm}|}{\textbf{A. Lanitis et al. \cite{Lanitis:2004:CDC:2225304.2226166}}} & 2004 & Private (40, 400) & Active Appearance Models & Quadratic Aging Function & N/A & $3.82\pm5.58$&\\[5pt] \hline
			\multicolumn{1}{|p{2.5cm}|}{\textbf{X. Geng et al. \cite{Geng:2006:LFA:1180639.1180711}}} & 2006 & FG-NET (82, 1.002) & AGES & Regression & N/A & $6.77$ &\\[5pt] \hline
		\end{tabular}
	}
	\caption{Age Estimation Methods}
	\label{tab:age-methods}
\end{table}