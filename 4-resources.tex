\chapter{Data Collection} \label{chap:data}

As described in Section \ref{sec:ageDB} there are many age-based databases of facial images. However, all existing datasets are based on real age estimation. 

\note{Re-write and organize}

In our proposed challenge, however we propose the first dataset to recognize the age people look like based on the opinion on many users using a new crowsourcing data collection and labeling application.

We developed a web application in order to collect and label an age estimation dataset online by the community. The application uses the Facebook API to facilitate the access and so reach more people with a broader background and also it allow us to easily collect data from the participants, such as gender, nationality and age. We show some panels of the application in the Figure 1(a), 1(b) and 1(c).
The web application was developed in a gamified way, i.e. the users or players get points for uploading and labeling images, the closer the age guess was to the apparent age (average labeled age) the more points the player obtains. In order to increase the engagement of the players we add a global and friends leaderboard where the users can see their position in the ranking. We ask the users to upload images of a single person and we give them tools to crop the image if necessary, we also ask them to give the real age (or as close as possible) of the person in uploaded image, allowing more analysis and
comparisons with real age and apparent age.

Few weeks after release the application we have alread collected near 1000 images and near 10000 votes. These numbers will continue growing in order to generate the future competition. Some of the properties of the database which is being collected with the web application are listed below:

\begin{itemize}
	\item Thousands of faces labeled by many users.
	\item Images with background.
	\item Non-controlled environments.
	\item Non-labeled faces neither landmarks, making the estima-
	tion problem even harder.
	\item One of the first datasets in the literature including
	estimated age labeled by many users to define the ground truth
	with the objective of estimating the age.
	\item The evaluation metric will be pondered by the mean and
	the variance of the labeling by the participants.
	\item The dataset also provides for each image the real age
	although not used for recognition (just for analysis purposes).
\end{itemize}

In the same way for all the labelers we have their nationality, age, and gender, which will allow analyzing demographic and other interesting studies among the correlation of labelers. In relation to the properties of existing datasets shown in Table I, ours include labels of the real age of the individuals and the apparent age given by the collected votes, both age distributions are shown in the Figure 2. The images of our database has been taken under very different conditions, which makes it more challenging for recognition purposes.

\section{Web Application}

\section{HuPBA Age Dataset}