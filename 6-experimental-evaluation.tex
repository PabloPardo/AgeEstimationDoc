\chapter{Results} \label{chap:experiments}

This Chapter describes the experiments done in this work and the results achieved.

\section{Datasets}
Two datasets were used in the experiments, our collected dataset HuPBA-AgeGuess and the classic benchmark database FG-NET.

\textbf{FG-NET} consists of 1002 frontal face images of 82 different individuals. The image quality varies a lot in the dataset since there are images in grey scale and RGB. The face position is frontal and under similar illumination conditions. The dataset also contain 68 facial landmarks for each face image.

\begin{figure}[!h]
	\centering
	\includegraphics[width=\textwidth]{figures/FGNET_sample}
	\caption{FG-NET image samples.}
	\label{fig:imgSample1}
\end{figure}

\textbf{HuPBA-AgeGuess} dataset (\ref{fig:imgSample2}) used in this project is a subset of the 4865 images filtered out by a minimum number of votes per image. This subset contains 3398 face images. The images are captured in the wild so the faces position vary up to $\pm90º$ and the illumination is also different in every picture.

\begin{figure}[!h]
	\centering
	\includegraphics[width=\textwidth]{figures/HuPBA_sample}
	\caption{HuPBA-AgeGuess image samples.}
	\label{fig:imgSample2}
\end{figure}

Given the characteristics of the used descriptors in this work, both databases could be incremented by computing the mirror image, doubling the number of faces of each dataset (i.e. 2004 faces for the FG-NET and 6796 face images for the HuPBA-AgeGuess dataset).

\section{Evaluation Metrics} 

The two most commonly used evaluation metrics in age estimation are \acrfull{mae} and \gls{cs}.

The \gls{mae} is described as

\begin{equation}
MAE = \frac{1}{N}\sum_i^{N} |e_i|
\end{equation}

where $e_i$ is the error of the $ith$ instance, i.e. $e_i = |\hat{y_i} - y_i|$ where $y_i $ is the real label and $\hat{y_i}$ is the predicted label. This metric tells the average number of years that the prediction is wrong.

\gls{cs} is defined as the percentage of test images such that the absolute error is not higher than a threshold, $t$ (in years). i.e.,

\begin{equation}
CS(t) = (1 - \frac{1}{N}\sum_i^N h(|\hat{y_i} - y_i| - t))\cdot 100
\end{equation}
\begin{equation}
h(x) = 
\begin{cases}
1,				& \text{if } x \geq 0\\
0,              & \text{otherwise}
\end{cases}
\end{equation}

where $y_i$ is the age label of the $ith$ test image and $\hat{y_i}$ is the age prediction of the $ith$ test image.

\section{Experimental Settings}
This section describes the experimental setup and the parameters used in the two proposed methods.

\subsection{Biologically Inspired Method}

As described in Section \ref{sec:BIF} a \gls{svm} classifier and three \gls{svr} regressors were trained for this method. 

In order to find the best parameters a grid search was performed to find the best parameters. The parameters that formed the search space were the ones required by the \gls{svm} and \glspl{svr} and its kernels, which are the following:

\begin{itemize}
	\item \textbf{Penalty term ($C$)}: This is the Support Vector parameter that deals with the cost of a misclassification over all the classification task. 
	
	The $C$ parameters tried in the search space were between 0.1 and 2. The best parameters in the Age Group Classification were between 1 and 2 and the best parameters for the \glspl{svr} were between 0.1 and 1.
	
	\item \textbf{Influence term ($\gamma$)}: It is the \gls{rbf} kernel parameter. It determines how far the influence of a single training example reaches.
	
	The $\gamma$ parameters used were between 0.001 and 1, being between 0.001 and 0.01 the ones with better performance.
\end{itemize}

In order to train and validate the parameters a nested 10-fold cross validation technique was used.


\subsection{Deep Learning Method}
10-fold cross validation


\section{Analysis of the Experiments}

\begin{figure}[!h]
	\centering
	\includegraphics[width=\textwidth]{figures/cum_score}
	\caption{Cummulative Score }
	\label{fig:cumS}
\end{figure}
